\section{Introduction}
\label{sec:intro}

There is a strong sentiment among the public and the popular media that polarization is increasing \parencite{pew-2014-political-polarization, kaysen-singer-2024-movers-polarization}.
Indeed, a recent poll revealed polarization as the second-most import issue to American voters, trailing only their economic concerns \parencite{nytimes-siena-2025-registered-voter-crosstabs}.
%
But is there really more disagreement among the American public on core ideological beliefs?
Or is polarization superficial with internal beliefs remaining similar?

A large literature documents sustained growth in what has been dubbed \emph{affective} polarization---defined as rising dislike for the other side of the political aisle \parencite{iyengar-lelkes-2019-affective-polarization, boxell-gentzkow-2023-cross-country-affective, boxell-gentzkow-2017-internet-polarization, iyengar-sood-lelkes-2012-affect, lelkes-sood-iyengar-2017-hostile-audience}.
A related line of work shows that \emph{ideological} polarization---defined as divergence on policy positions and other political beliefs---has similarly increased among congresspeople and other political elites over the same time period \parencite{mccarty-poole-rosenthal-2006-polarized-america, elsas-fiselier-2023-elite-polarization-dimensions, knoll-2024-elite-polarization-boon-bane}.

However, trends in ideological polarization among the mass public are less well understood.
Are everyday Americans becoming more separated in their core ideological, moral, and political beliefs?
Some work argues that this separation has occurred only among elites with the electorate remaining similarly (un)divided \parencite{fiorina-abrams-2011-culture-war, lelkes-2016-mass-elite-polarization}.
Others claim meaningful growth in ideological divisions.
For example, \textcite{ojer-etal-2025-multidimensional-polarization} embed survey-respondents opinions in a two-dimensional latent policy space and take the increasing distance between Democrats and Republicans positions as evidence of increasing party polarization.
\textcite{abramowitz-saunders-2008-polarization-myth} cite the increasing presence of consistently liberal or consistently conservative opinions across multiple issues.

One reason for this disagreement is the lack of a consistent definition for what constitutes ideological polarization.
This paper contributes such a definition.
I start with the observation that the mere prevalence of extreme opinions does not, by itself, constitute polarization.
What matters is opinion dispersion and---once multiple issues are considered---the structure of that dispersion across issues.
Intuitively, polarization is maximized when views are concentrated at opposing extremes with little mass in between and when opinions are highly correlated across many issues.
Formally, I summarize the joint distribution of policy opinions using the covariance matrix of survey responses and use their matrix norms as scalar indices of polarization.
My preferred index is the \emph{spectral norm} (the largest eigenvalue\footnote{
	Because covariance matrices are positive semidefinite, the spectral norm and spectral radius coincide.
	This is not the case in general.
}), which admits an intuitive interpretation as the variance of the first principal component.\footnote{
	The same framework accommodates alternative matrix norms (trace/nuclear, Frobenius), which can be interpreted as different ways of aggregating the variance of each of the principal components.
	My main results are robust to norm choice.
}
Using this measure, I find that polarization has increased on most---but not all---topics over the past three decades.

One strength of the spectral radius as a measure of polarization is that it admits two transparent decompositions that allow me to partition both its levels and changes.
First, I decompose polarization into a term capturing the degree to which people disagree in general and a term capturing the extent to which opinions are correlated across issues.
I find that---with the notable exception of race-related issues---increases in polarization have been driven mostly by increases in general disagreement and \emph{not} by dimensional collapse.

Second, I examine the degree to which increases in polarization are explained by divergence in opinions among political parties and other demographic groups.
For example, has polarization increased simply because of divergence in the opinions of Democrats and Republicans?
I find a nuanced answer to this question.
Polarization within political party is as as high and in many cases even higher than overall polarization.%
\footnote{
	For example, Republicans are consistently more internally polarized on spending issues than the general public and Democrats are internally polarized on police and race-related issues.
}
Furthermore, I find that the driver of increasing polarization varies greatly depending on the topic at hand.
For example, increasing polarization on race and welfare issues has been driven by between-party changes whereas increasing polarization about law-enforcement has been driven almost entirely by changes within political party.
Additionally, I find that differences of opinion between demographic groups (e.g., gender, race, geography, education, religion) in general explain a very small proportion of the observed levels of and trends in polarization.
Together, these results question existing literature that suggests increasing polarization is driven by the clustering of opinions within demographic niches \parencite{gennaioli-tabellini-2021-identity-beliefs}.

Concretely, this paper makes three contributions:
\begin{enumerate}
	\item
	      First it motivates and develops a covariance-based measure of ideological polarization that takes into account both disagreement on individual issues and the joint distribution of opinions across many issues.
	      % This method delivers a single scalar measure that reduces to variance in the case of a single issue.\note{too much ``issue''}
	\item
	      Second, it develops two decompositions of changes in polarization.
	      The first partitions changes into a term capturing the degree to which opinions project onto a single ``us–vs–them'' axis and a term capturing the strength of disagreement in general.
	      The second decomposes changes into within- and between-group components.
	\item
	      Third, it offers a unified portrait of mass ideological polarization over the past three decades across several topic domains by applying the methods from the previous two bullets to the University of Chicago NORC's General Social Survey \parencite{gss}.
	      We find that (1) polarization has increased, (2) these increases have been mostly driven by increases in general disagreement and \emph{not} by dimensional collapse, and (3) these increases are not entirely explained by diverging party positions or diverging positions of demographic groups like gender, race, geography, or religion.
\end{enumerate}

The remainder of the paper proceeds as follows:
\cref{sec:theory} introduces my polarization measure and presents the statistical framework for the remainder of the paper;
\cref{sec:decompositions} derives and explains my two decompositions;
\cref{sec:gss} applies these techniques to the GSS and reports results;
\cref{sec:conclusion} concludes and outlines potential for future work.

