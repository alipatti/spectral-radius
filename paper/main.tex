\documentclass[bib,theorems,pgf]{./paper/ali-paper}

\RequirePackage{./paper/ali-macros}
\addbibresource{./paper/refs.bib}

\title{Public Ideological Polarization}

\author{
	Alistair Pattison%
	\footnote{Harvard University and Opportunity Insights (apattison@g.harvard.edu).}
}
\date{November 2025}

% include header if compiling locally, but not on arxiv version
\IfFileExists{.texlabroot}{
	\pagestyle{fancy}
	\fancyhead{}
	\chead{
		\footnotesize
		\color{red!60!white}
		This pdf was compiled on \today. \\
		Please see \href{https://github.com/alipatti/spectral-radius}{\tt github.com/alipatti/spectral-radius} for the complete and most recent version of this working paper.
	}
	\renewcommand{\headrulewidth}{0pt}
}{}

% spacing
\linespread{1.5}

\begin{document}

\maketitle

\begin{abstract}
	This paper provides a novel summary measure of ideological polarization in the American public based on the joint distribution of survey responses.
	Intuitively, polarization is maximized when views are concentrated at opposing extremes with little mass in between and when opinions are highly correlated across many issues.
	Using this measure, I show that public polarization has been increasing for the past three decades and that these changes are mostly due to increases in general disagreement, not dimensional collapse.
	Furthermore, these increases take place largely \emph{within} group; they are not explained by the diverging opinions of Democrats and Republicans, nor divergence of opinions across gender, geography, education, or any other demographic divide.
\end{abstract}

% main content
\section{Introduction}
\label{sec:intro}

There is a strong sentiment among the public and the popular media that polarization is increasing \parencite{pew-2014-political-polarization, kaysen-singer-2024-movers-polarization}.
Indeed, a recent poll revealed polarization as the second-most import issue to American voters, trailing only their economic concerns \parencite{nytimes-siena-2025-registered-voter-crosstabs}.
%
But is there really more disagreement among the American public on core ideological beliefs?
Or is polarization superficial with internal beliefs remaining similar?

A large literature documents sustained growth in what has been dubbed \emph{affective} polarization---defined as rising dislike for the other side of the political aisle \parencite{iyengar-lelkes-2019-affective-polarization, boxell-gentzkow-2023-cross-country-affective, boxell-gentzkow-2017-internet-polarization, iyengar-sood-lelkes-2012-affect, lelkes-sood-iyengar-2017-hostile-audience}.
A related line of work shows that \emph{ideological} polarization---defined as divergence on policy positions and other political beliefs---has similarly increased among congresspeople and other political elites over the same time period \parencite{mccarty-poole-rosenthal-2006-polarized-america, elsas-fiselier-2023-elite-polarization-dimensions, knoll-2024-elite-polarization-boon-bane}.

However, trends in ideological polarization among the mass public are less well understood.
Are everyday Americans becoming more separated in their core ideological, moral, and political beliefs?
Some work argues that this separation has occurred only among elites with the electorate remaining similarly (un)divided \parencite{fiorina-abrams-2011-culture-war, lelkes-2016-mass-elite-polarization}.
Others claim meaningful growth in ideological divisions.
For example, \textcite{ojer-etal-2025-multidimensional-polarization} embed survey-respondents opinions in a two-dimensional latent policy space and take the increasing distance between Democrats and Republicans positions as evidence of increasing party polarization.
\textcite{abramowitz-saunders-2008-polarization-myth} cite the increasing presence of consistently liberal or consistently conservative opinions across multiple issues.

One reason for this disagreement is the lack of a consistent definition for what constitutes ideological polarization.
This paper contributes such a definition.
I start with the observation that the mere prevalence of extreme opinions does not, by itself, constitute polarization.
What matters is opinion dispersion and---once multiple issues are considered---the structure of that dispersion across issues.
Intuitively, polarization is maximized when views are concentrated at opposing extremes with little mass in between and when opinions are highly correlated across many issues.
Formally, I summarize the joint distribution of policy opinions using the covariance matrix of survey responses and use their matrix norms as scalar indices of polarization.
My preferred index is the \emph{spectral norm} (the largest eigenvalue\footnote{
	Because covariance matrices are positive semidefinite, the spectral norm and spectral radius coincide.
	This is not the case in general.
}), which admits an intuitive interpretation as the variance of the first principal component.\footnote{
	The same framework accommodates alternative matrix norms (trace/nuclear, Frobenius), which can be interpreted as different ways of aggregating the variance of each of the principal components.
	My main results are robust to norm choice.
}
Using this measure, I find that polarization has increased on most---but not all---topics over the past three decades.

One strength of the spectral radius as a measure of polarization is that it admits two transparent decompositions that allow me to partition both its levels and changes.
First, I decompose polarization into a term capturing the degree to which people disagree in general and a term capturing the extent to which opinions are correlated across issues.
I find that---with the notable exception of race-related issues---increases in polarization have been driven mostly by increases in general disagreement and \emph{not} by dimensional collapse.

Second, I examine the degree to which increases in polarization are explained by divergence in opinions among political parties and other demographic groups.
For example, has polarization increased simply because of divergence in the opinions of Democrats and Republicans?
I find a nuanced answer to this question.
Polarization within political party is as as high and in many cases even higher than overall polarization.%
\footnote{
	For example, Republicans are consistently more internally polarized on spending issues than the general public and Democrats are internally polarized on police and race-related issues.
}
Furthermore, I find that the driver of increasing polarization varies greatly depending on the topic at hand.
For example, increasing polarization on race and welfare issues has been driven by between-party changes whereas increasing polarization about law-enforcement has been driven almost entirely by changes within political party.
Additionally, I find that differences of opinion between demographic groups (e.g., gender, race, geography, education, religion) in general explain a very small proportion of the observed levels of and trends in polarization.
Together, these results question existing literature that suggests increasing polarization is driven by the clustering of opinions within demographic niches \parencite{gennaioli-tabellini-2021-identity-beliefs}.

Concretely, this paper makes three contributions:
\begin{enumerate}
	\item
	      First it motivates and develops a covariance-based measure of ideological polarization that takes into account both disagreement on individual issues and the joint distribution of opinions across many issues.
	      % This method delivers a single scalar measure that reduces to variance in the case of a single issue.\note{too much ``issue''}
	\item
	      Second, it develops two decompositions of changes in polarization.
	      The first partitions changes into a term capturing the degree to which opinions project onto a single ``us–vs–them'' axis and a term capturing the strength of disagreement in general.
	      The second decomposes changes into within- and between-group components.
	\item
	      Third, it offers a unified portrait of mass ideological polarization over the past three decades across several topic domains by applying the methods from the previous two bullets to the University of Chicago NORC's General Social Survey \parencite{gss}.
	      We find that (1) polarization has increased, (2) these increases have been mostly driven by increases in general disagreement and \emph{not} by dimensional collapse, and (3) these increases are not entirely explained by diverging party positions or diverging positions of demographic groups like gender, race, geography, or religion.
\end{enumerate}

The remainder of the paper proceeds as follows:
\cref{sec:theory} introduces my polarization measure and presents the statistical framework for the remainder of the paper;
\cref{sec:decompositions} derives and explains my two decompositions;
\cref{sec:gss} applies these techniques to the GSS and reports results;
\cref{sec:conclusion} concludes and outlines potential for future work.


\section{Measuring Polarization}

On one single issue, polarization can be characterized by some measure of scale (e.g. variance) or tail heaviness (e.g. kurtosis).

For binary data, variance corresponds to 

in the real world, we don't observe latent ``polarization''

need to generalize this to multiple

"locked in a zero-sum struggle along a single “us-vs-them” dimension"
% https://d1y8sb8igg2f8e.cloudfront.net/documents/The_Case_for_Fusion_Voting_and_a_Multiparty_Democracy_in_America_2022.pdf

\begin{figure}
	\centering
	\label{fig:norm-comparison}
	\begin{minipage}{.5\textwidth}
		\includegraphics[width=\textwidth]
			{./figures/examples/motivation/norm_illustration.pdf}%
	\end{minipage}%
	\hspace{.1\textwidth}
	\begin{tabular}{r l | c | c}
		\multicolumn{2}{c|}{Norm} & Expression & $p$ \\
		\hline
		Spectral  & $\norm \Sigma_2$ & $\max(a, b)$       & $\infty$ \\ 
		Nuclear   & $\norm \Sigma_*$ & $a + b$            & $1$ \\ 
		Frobenius & $\norm \Sigma_F$ & $\sqrt{a^2 + b^2}$ & $2$ \\ 
	\end{tabular}
	\caption{
		The left-hand panel shows the ellipse $\set{ \mathbf x \trans \Sigma \mathbf x : \norm{\mathbf x} = 1 }$ induced by the matrix $\Sigma = \operatorname{diag}(2, 1)$.
		The right-hand panel shows how the various matrix norms can be expressed as $p$-norms of the the matrix's eigenvalue vector, $(a, \, b) \trans$.
	}
\end{figure}

\begin{figure}
	\centering
	\label{fig:gss-example}
	\includegraphics{./figures/examples/motivation/example.pdf}
	\caption{
		TODO: write 
		% \todo{make axis labels smaller}
	}
\end{figure}

\subsection{Setup and Definition}

Let $X_1,\dots,X_n \in \mathbb{R}^p$ be i.i.d.\ observations from some $p$-dimensional distribution with mean $0$ and covariance matrix $\Sigma$.
Define the sample covariance matrix $\widehat S_n \ceq \frac 1n \sum_i X_i X_i \trans$.
\todo{do we want $n-1$ in denom?}

Let $\lambda_1 \geq \lambda_2 \geq \cdots \geq \lambda_p \geq 0$ be the eigenvalues of $\Sigma$ and $\hat\lambda_1 \ge \hat\lambda_2 \ge \cdots \ge \hat\lambda_p$ the eigenvalues of $\widehat S_n$.

Define $r = \rho(\Sigma)$ to be the spectral radius of $\Sigma$ and $\hat r_n = \rho(\widehat S_n)$ the spectral radius of the sample covariance matrix.

There are several ways to think about this
\begin{itemize}
	\item the maximum eigenvalue of the covariance matrix,
	\item the maximum eigenvalue of the covariance matrix,
	\item the proportion of variance explained by the first principal component,
	\item oblong-ness of the data elipsoid
\end{itemize}

Advantages of this over other approaches
\todo{citations}
\begin{itemize}
	\item it's model-free (e.g. doesn't require ideal point model, makes no distributional requirements of the data)
	\item easily allows imposition of a model via the covariance matrix estimation if that's desired (e.g. sparsity structure of covariance matrix, shrinkage of covariance matrix)
	\item allows us to tap into PCA/random matrix theory
	\item reduces to well-understood measure in one dimension (variance)
	\item computationally tractable
	\item gracefully handles missing data (very common in surveys) \todo{link to where we explain this in the application section}
\end{itemize}

Disadvantages
\todo{citations}
\begin{itemize}
	\item different sets of questions are incomparable
\end{itemize}

\subsection{Relationship to Polarization in a Single Dimension}

There is a two-way relationship between

The first proposition follows immediately from definition (one-by-one matrices are trivially diagonalized). The second

\begin{proposition}
\end{proposition}


\begin{proposition}
	\label{thm:one-dim-implies-multiple-dim}
	Let $x \sim \mathcal F(a)$ be a person's latent one-dimensional position modeled as random variable with finite variance $a \in \R$.
	Let $\beta \in \R^d$ be the sensitivity of each of issue to this latent position such that her revealed positions with respect to individual policies are
	\begin{equation}
		y = \beta x + \epsilon;
		\hspace{1in}
		\Var(\epsilon) = \Gamma \in \R^{d \times d}.
	\end{equation}
	Let $\Sigma = \Var(y)$ and $r = \rho(\Sigma)$.
	Then, $r$ is non-decreasing with respect to $a$.
\end{proposition}

Note that this proposition is quite flexible: we make no assumptions about the latent distribution other than it has a finite second moment. The error term is also allowed to take any form---in particular it can have complex non-diagonal covariance structure.

We first establish a few lemmas that greatly simiplify the proof. The first is regarding the spectral radius of a positive rank-one update to a positive semidefinite matrix, which we prove as conesequence of Weyl's inequalities \cite{weyl-1912-inequalities}. This result is very similar in flavor those of \todo{section number?} \textcite{golub-1973-eigenvalues}. We then leverage this to prove our proposition and close with a discussion of two suffcient conditions under which the phrase "non-decreasing" on the last line of \autoref{thm:one-dim-implies-multiple-dim} becomes "strictly increasing".

\begin{lemma}
	\label{lem:rank-one-update}
	If $D \in \R^{n \times n}$ is symmetric and positive semidefinite, $v \in \R^n$, and $c \geq 0$, then
	\begin{equation}
		\rho(c v v \trans + D) \geq \rho(D).
	\end{equation}
\end{lemma}

\begin{proof}

	See \textcite[\S 5]{golub-1973-eigenvalues}.
	% \todo{do we want to spell out the proof from Weyl's inequalities?
	% https://nhigham.com/2021/03/09/eigenvalue-inequalities-for-hermitian-matrices/
\end{proof}

\begin{lemma}
	\label{lem:rank-one-update-increasing}
	Let the setup be the same as in \autoref{lem:rank-one-update}. Then $\rho(c v v \trans + D)$ is non-decreasing with respect to $c$.
\end{lemma}

\begin{proof}
	Let $\epsilon > 0$. Then,
	\begin{equation}
		\begin{aligned}
			\rho((c + \epsilon) v v \trans + D)
			\ = \ \rho(\epsilon v v \trans + (c v v \trans + D))
			\ \geq \ \rho(c v v \trans + D)
		\end{aligned}
	\end{equation}
	by \autoref{lem:rank-one-update} because $c v v \trans + D \succeq 0$ by the closure of positive semidefinite matrices under addition.
\end{proof}

Proof of the original result is now quite simple:

\begin{proof}[Proof of \autoref{thm:one-dim-implies-multiple-dim}]
	The spectral radius of $\Sigma$ is
	\begin{equation}
		\rho(\Sigma)
		\ = \ \rho(\Var(\beta x + \epsilon))
		\ = \ \rho(a \beta \beta \trans + \Gamma)
	\end{equation}
	Because $\Gamma$ is a covariance matrix, it's symmetric and positive semidefinite and \autoref{lem:rank-one-update-increasing} gives us that $\rho(\sigma)$ is non-decreasing in $a$ as desired.
\end{proof}

Strict increase if $\beta$ nontrivially projects onto the first eigenspace of $\Gamma$. This happens on all but a measure zero subset of possible $(\beta, \Gamma)$ pairs.
\todo{confim this}

\begin{proposition}
	If $a = \Var(x) > \rho(\Gamma) = r$, then $r$ is \emph{strictly increasing} with respect to $a$.
\end{proposition}

\begin{corollary}
	There exists some $a > 0$ after which $r$ is strictly increasing.
\end{corollary}

\subsection{Estimating the Spectral Radius}

Can leverage a lot of PCA theory going back to \textcite{anderson-1963-pca-asymptotics}. For an overview, see \textcite{jolliffe-2002-pca} or \textcite{zagidullina-2021-random-matrix-theory}.

\subsubsection{Asymptotics}

See \textcite{anderson-2003-multivariate} for more.

\begin{proposition}[Consistency of sample eigenvalues]
	The sample eigenvalues $\hat \lambda_i$ are consistent estimates of the population eigenvalues $\lambda_i$.
\end{proposition}

\begin{proof}
	We prove the stronger notion of almost-sure convergence. The (strong) law of large numbers gives almost-sure convergence of the covariance matrix $\widehat S_n \overset{a.s.}{\to} \Sigma$. The set of eigenvalues is a continuous function of a matrices entries \cite[Ch. 2, Th. 5.14, p. 118]{kato-1980-perturbation}, so the continuous mapping theorem yields that $\hat \lambda_i \overset{a.s.}{\to} \lambda_i$.
\end{proof}

\begin{proposition}[Asymptotic normality of sample eigenvalues]
	Let $X_1,\dots,X_n \in \mathbb{R}^p$ be i.i.d.\ with mean $0$ and covariance $\Sigma$.
	If $\Sigma$ has eigenvalues $\lambda_1 \geq \lambda_2 \geq \cdots \geq \lambda_p \geq 0$ with corresponding sample eigenvalues $\hat\lambda_1 \ge \hat\lambda_2 \ge \cdots \ge \hat\lambda_p$, then
	\begin{equation}
		\sqrt{n} \paren{\hat\lambda_i - \lambda_i}
		\overset{\tiny d}{\longrightarrow}
		N\!\paren{0,\;\sigma_i^2},
		\qquad i=1,\dots,p.
	\end{equation}
	If $\set{X_i}$ are normally distributed, then $\sigma_i^2 = 2\lambda_i^2$. If not,
	$\sigma_i^2$ depends (somewhat complexly) on the fourth cumulants of $X_i$.
\end{proposition}

The normal result is due to \textcite{anderson-1963-pca-asymptotics}. The non-normal case is due to \textcite{waternaux-1976-nonnormal-eigenvalues} and expanded upon by \textcite{eaton-1991-eigenvalues}. The non-normal case will be of most use to us because most survey data is decidedly not normal (e.g. binary response or multiple choice). \note{We'll probably have to bootstrap, but can take advantage of normality to produce standard errors instead of using the boostrap quantiles.}

From these facts, it follows trivially that the largest sample eigenvalue $\hat \lambda_1$ is a consistent and asymptotically-normal estimator of the spectral radius.

\subsubsection{Finite-Sample Overestimation}

Finite sample estimates for $\hat \lambda_1$ will overestimate

\todo{shrinkage?}


\section{Decompositions}
\label{sec:decompositions}

\subsection{First Principal Component}
\label{sec:trace-concentration}

The fact that the spectral norm of the covariance matrix can be understood as the variance of the first principal component means that $\norm \Sigma _2$ admits a straightforward multiplicative decomposition as the total variance ($\tr \Sigma$) times the explanatory power of the first principal component ($\lambda_1 / \tr \Sigma$):
\begin{equation}
	\norm{\Sigma}_2
	\ = \ \lambda_1
	\ = \ \tr \Sigma \cdot \lambda_1 / \tr \Sigma.
\end{equation}
We call the first component the total variance and the second component the ``spectral concentration''.

By itself, this decomposition is somewhat useful in telling us how much variation projects onto a single dimension, but it's real utility comes in unpacking the source of the difference between two distributions' polarization.
For example, suppose that there are two distributions $\mathcal F_0$ and $\mathcal F_t$ with covariance matrices $\Sigma_0$ and $\Sigma_t$ respectively.
Then, we can decompose the percentage difference in their polarizations as the change in total variance times the change in spectral concentration:
\begin{equation}
	\label{eq:percent-change-trace-concentration}
	\frac{\norm{\Sigma_1}_2}{\norm{\Sigma_0}_2}
	\ = \ \frac{\tr \Sigma_1}{\tr \Sigma_0} \cdot \frac{\lambda_{0,1} / \tr \Sigma_0}{\lambda_{t,1} / \tr \Sigma_t}
\end{equation}

\Cref{fig:stretch-v-scale} illustrates this idea, and in \cref{sec:trace-concentration-decomp} discusses results from applying this technqiue to survey data to decompose changes in the polarization of the American public over time.
We find that most changes are due to changes in the total variance term.

\begin{figure}
	\centering
	\caption{\XX}
	\label{fig:stretch-v-scale}
	\includegraphics{./figures/examples/decomps/stretch_v_scale/stretched.pdf}%
	\hspace{.5in}
	\includegraphics{./figures/examples/decomps/stretch_v_scale/scaled.pdf}%
	\notes{
		The spectral norm can be multiplicatively decomposed into the product of the spectral concentration (i.e., the proportion of variance explained by the first principal component) and the trace of the covariance matrix (i.e., the sum of the individual variances).
		In this figure, the blue distributions are the same in both panels and the green distributions share the same larger spectral radius.
		However, the reasons for this increase are quite different.
		In the left-hand panel, the larger norm is entirely due to an increase of the relevance of the first principal component with the total variance of the two distributions held constant.
		In the right-hand panel, the difference is entirely due to the total variance with a fixed spectral concentration.
	}
\end{figure}

\begin{figure}
	\centering
	\caption{\XX}
	\label{fig:between-group-sign}
	\includegraphics{./figures/examples/decomps/by_group/negative.pdf}%
	\includegraphics{./figures/examples/decomps/by_group/zero.pdf}%
	\includegraphics{./figures/examples/decomps/by_group/positive.pdf}
	\notes{
		The sign and magnitude of of the between-group polarization $\rho_b$ can vary greatly depending on the means and covariance structure of the groups.
		In the left-hand panel, the individual subgroups (dashed) are distributed such that the within-group polarization is \emph{higher} than the pooled (black) polarization, i.e., $\rho_b < 0$.
		In the center panel, the between group polarization is orthogonal to the within-group polarization and $\rho_b = 0$.
		The rightmost panel shows a scenario where the within- and between-group polarizations align to yield $\rho_b > 0$.
	}
\end{figure}

\subsection{Between vs. Within Groups}
\label{sec:between-v-within}

The following section explores how polarization can come from both disagreement \emph{within} a particular group (e.g. disagreement among Democrats about immigration) and disagrement \emph{between} groups (e.g. Democrats and Republicans hold, on average, quite different opinions about law enforcement).
To start, we ammend our statistical setup to allow for such groupings.
%
In particular, let
\begin{equation}
	z_i \sim \operatorname{Multinomial}(p_1, \ldots, p_G)
\end{equation}
be an integer $z_i \in \set{1, \ldots, G}$ indicating the group to which individual $i$ belongs.
We allow each group to draw their opinions from an entirely different distribution, so the revealed opinions of indivual $i$ on each of the $j$ issues is given by
\begin{equation}
	x_{ij} = \sum_g \mathbf 1 (z_i = g) \cdot x_{ijg}
\end{equation}
where $x_{ijg}$ is the potential opinion of individual $i$ on issue $j$ were they a member of group $g$.
Then, the multidimensional analogue to he law of total variance \parencite{TODO} allows us to decompose the covariance matrix of $\mathbf x_i = (x_{i1}, \ldots, x_{ip})$ into a within- and between-group component:
\begin{equation}
	\begin{aligned}
		\label{eq:within-between-decomposition-of-sigma}
		\Sigma
		 & = \Var(\mathbf x_i)                                                         \\
		 & = \E{\Var(\mathbf x_i \given z_i)} + \Var\paren{\E{\mathbf x_i \given z_i}} \\
		 & = \underbrace{\sum_g p_g \Sigma_g}_{\substack{\text{within-group}           \\ \text{components}}} \ + \ \underbrace{\sum_g p_g \paren{\mu_g - \mu}\paren{\mu_g - \mu} \trans}_{\text{between-group component}}
	\end{aligned}
\end{equation}
where $\mu_g = \E{\mathbf x_{i} \given z_i = g}$ is the mean vector of group $g$ and $\Sigma_g = \Var(\mathbf x_i \given z_i = g)$ is the group's covariance matrix, and $\mu$ is the overall mean.
%
Unfortunately, the spectral norm is not linear, so this does not immediately yield a within- and between-group decomposition of $\norm \Sigma _2$.
Fortunately, it is \emph{sub-linear}.\footnote{Meaning $\norm{A + B}_2 \leq \norm A_2 + \norm B_2$, i.e., it satisfies the triangle inequality.}.

If we let $\Sigma_w$ and $\Sigma_b$ denote the within- and between-group components of the decomposition in \autoref{eq:within-between-decomposition-of-sigma}, the triangle inequality for norms lets us write $\norm{\Sigma}_2$ as the sum of $\norm{\Sigma_w}_2$ and $\norm{\Sigma_b}_2$, less an additional slack term $s_b \geq 0$ that captures the tightness of the triangle inequality for this particular pair of matrices:
\begin{equation}
	\label{eq:norm-sigma-decomposition-1}
	\norm{\Sigma}_2 = \norm{\Sigma_w}_2 + \norm{\Sigma_b}_2 - s_b.
\end{equation}
In general, the interpretation of $s_b$ varies depending on vector space and the norm being used,\footnote{
	For example, with the $\mathcal L_2$ norm on $\R^n$, the slack is related to the angle between the two vectors.
} but in our spectral-norm setup, $s_b$ captures the degree to which the principal eigenvectors of $\Sigma_w$ and $\Sigma_b$ align, with $s_b$ vanishing precisely when the principal eigenvectors collide (or when one of the matrices is zero).
In our setting, this is means that $s_b$ captures the difference in the direction of polarization between $\Sigma_w$ and $\Sigma_b$:
Small $s_b$ means that groups disagree along a similar axis amongst themselves as the disagreement between groups.
Large $s_b$ means that the directions of polarization are different.

However, we are often concerned not just with the difference in the direction of polarization between $\Sigma_b$ and $\Sigma_w$, but the differences in the directions of polarization among the groups themselves.
Do Democrats disagree in the same way on race issues as Republicans do?
To get at this, we can further decompose $\norm{\Sigma_2}_2$ into the norms of its individual group-specific covariance matrices plus another slack term $s_b \geq 0$:
\begin{equation}
	\norm{\Sigma}_2 = \sum_i p_i \norm{\Sigma_i}_2 + s_w - \norm{\Sigma_b}_2 - s_b.
\end{equation}
Here, $s_w$ captures the degree to which each group is polarized on a different set of issues in the same way that $s_b$ in \autoref{eq:norm-sigma-decomposition-1} above captured how the direction of polarization differed between $\Sigma_w$ and $\Sigma_b$.
For simplicity of presentation, we typically combine both slack terms and the between-group polarization into a single ``between-group'' category:
\begin{equation}
	\label{eq:norm-sigma-decomposition-final}
	\norm{\Sigma}_2
	\ = \ \underbrace{\sum_i p_i \norm{\Sigma_i}_2}_{\rho_{\text{within}}} + \underbrace{\norm{\Sigma_b}_2 - s_w - s_b}_{\rho_{\text{between}}}.
\end{equation}

The payoff from all this math is that both terms in \autoref{eq:norm-sigma-decomposition-final} have an intuitive interpreation:
\begin{itemize}
	\item
	      $\rho_w$ measures the extent to which each group is polarized amongst themselves and is simply the weighted average of the group-specific polarizations.
	\item
	      $\rho_b$ measures the extent to which there is polarization across groups. The norm of the between-group covariance matrix captures increased polarization coming from different groups holding different positions.
	      The slack terms accounts for the fact that overall polarization is diminished if different groups are divided on different issues.
	      Note that $\rho_b$ is \emph{not necessarily positive}—it can be (and often is the case) that the differing directions of polarization in individual groups ``cancel out'' the between group polarization.
	      \Cref{fig:between-group-sign} visualizes how between-group differences can amplify, offset, or leave unchanged overall polarization depending on the axis along which each subgroup is polarized.
\end{itemize}

We present results from applying this decomposition to the GSS in \cref{sec:party-decomp}.


\section{Application in the General Social Survey}

\subsection{Data}

See \autoref{sec:gss-questions} for the questions.

\subsection{Methods}


\subsubsection{Covariance Estimation}

\subsubsection{Standard Errors}

\subsection{Results}

\begin{landscape}
	\begin{figure}
		\centering
		% \input{figure.pgf}
		\caption{example figure}
	\end{figure}
\end{landscape}


\subsection{Decomposition}
\label{sec:first-pc-decompositions}

\subsection{Decompositions by Group}
\label{sec:group-decompositions}

Simply plug-in our covariance matrix estimators into \autoref{eq:norm-sigma-decomposition-final} and calculate $s$ as the residual term.

\section{Conclusion}

Introduced and motivated a new measure of polarization in the general public that is broadly applicable to all sorts of data.
Showed several decomposition of measure with intuitive interpretations.

Showed using data from the General Social Survey that polarization in the general public has increased over the last three decades across a wide variety of topics.
This is contrary to general poli sci literature which generally holds that polarization has only increased among elites (e.g. politicians).

These changes are being driven by a variety of factors, both within and across political parties.


\subsection{Future Work}

Explore wider variety of topics.
Would loved to have looked at immigration, but GSS doesn't provide good longitudinal info.

Many subgroup cuts are endogenous (e.g. political party, idology).
Would be nice to use a panel dataset to fix characteristics at t=0.

Use other data sets.

Look at political elites, e.g. congresspeople.
See how elite polarization correlates with constituent polarization.
Use LLM to produce panel of responses to many survey questions.
Decompose into entry/exit vs. changing opinions of people within.

Look at the effect of some sort of exogenous shock on polarization.
Geographic variation.

Take into account clustering rather than just spread \parencite{esteban-ray-1994-measurement-polarization, duclos-esteban-2004-polarization-concepts}.


% end matter
\newpage

\printbibliography

\newpage

\appendix

\section{Deferred Proofs}
\label{sec:deferred-proofs}

{\color{red}\textbf{Note:} this section is under construction --- some proofs lack detail.}

\subsection{Spectral Radius Estimator}

Let $\lambda_1 \geq \lambda_2 \geq \cdots \geq \lambda_p \geq 0$ be the eigenvalues of $\Sigma$ and $\hat\lambda_1 \ge \hat\lambda_2 \ge \cdots \ge \hat\lambda_p$ the eigenvalues of $\widehat S_n$.

To do so, I leverage the fact that our matrix norm can be expressed in terms of the covariance eigenvalues and lean heavily on PCA theory going back to \textcite{anderson-1963-pca-asymptotics}.
For an overview, see \textcite{jolliffe-2002-pca} or \textcite{zagidullina-2021-random-matrix-theory}.
See \textcite{anderson-2003-multivariate} for more.

\begin{proposition}[Consistency of sample eigenvalues]
	\label{prop:sample-eigenvalue-consistency}
	The sample eigenvalues $\hat \lambda_i$ are consistent estimates of the population eigenvalues $\lambda_i$.
\end{proposition}

\begin{proof}
	I prove the stronger notion of almost-sure convergence. The (strong) law of large numbers gives almost-sure convergence of the covariance matrix $\widehat S_n \overset{a.s.}{\to} \Sigma$. The set of eigenvalues is a continuous function of a matrices entries \cite[Ch. 2, Th. 5.14, p. 118]{kato-1980-perturbation}, so the continuous mapping theorem yields that $\hat \lambda_i \overset{a.s.}{\to} \lambda_i$.
\end{proof}

\begin{proposition}[Asymptotic normality of sample eigenvalues]
	\label{prop:sample-eigenvalue-normality}
	Let $X_1,\dots,X_n \in \mathbb{R}^p$ be i.i.d.\ with mean $0$ and covariance $\Sigma$.
	If $\Sigma$ has eigenvalues $\lambda_1 \geq \lambda_2 \geq \cdots \geq \lambda_p \geq 0$ with corresponding sample eigenvalues $\hat\lambda_1 \ge \hat\lambda_2 \ge \cdots \ge \hat\lambda_p$, then
	\begin{equation}
		\sqrt{n} \paren{\hat\lambda_i - \lambda_i}
		\overset{d}{\longrightarrow}
		N\!\paren{0,\;\sigma_i^2},
		\qquad i=1,\dots,p.
	\end{equation}
	If $\set{X_i}$ are normally distributed, then $\sigma_i^2 = 2\lambda_i^2$. If not,
	$\sigma_i^2$ depends (somewhat complexly) on the fourth cumulants of $X_i$.
\end{proposition}

\begin{proof}
	The normal result is due to \textcite{anderson-1963-pca-asymptotics}.
	The non-normal case is due to \textcite{waternaux-1976-nonnormal-eigenvalues} and expanded upon by \textcite{eaton-1991-eigenvalues}.
	The non-normal case will be of most use to us because most survey data is decidedly not normal (e.g. binary response or multiple choice).
\end{proof}

Asymptotic normality and consistency of the spectral radius estimator follow trivially from \cref{prop:sample-eigenvalue-consistency} and \cref{prop:sample-eigenvalue-normality}.

\subsection{Latent Model}

I first establish a few lemmas that greatly simplify the proof.
The first is regarding the spectral radius of a positive rank-one update to a positive semidefinite matrix, which I prove as consequence of Weyl's inequalities \parencite{weyl-1912-inequalities,tao-2010-eigenvalue-blog}.
This result is very similar in flavor those of \textcite{golub-1973-eigenvalues}.
I then leverage this to prove our proposition and close with a discussion of two sufficient conditions under which the phrase "non-decreasing" on the last line of \cref{thm:one-dim-implies-multiple-dim} becomes "strictly increasing".

\begin{lemma}
	\label{lem:rank-one-update}
	If $D \in \R^{n \times n}$ is symmetric and positive semidefinite, $v \in \R^n$, and $c \geq 0$, then
	\begin{equation}
		\rho(c v v \trans + D) \geq \rho(D).
	\end{equation}
\end{lemma}

\begin{proof}
	See \textcite[\S 5]{golub-1973-eigenvalues}.
	% \todo{do I want to spell out the proof from Weyl's inequalities?
	% https://nhigham.com/2021/03/09/eigenvalue-inequalities-for-hermitian-matrices/
\end{proof}

\begin{lemma}
	\label{lem:rank-one-update-increasing}
	Let the setup be the same as in \cref{lem:rank-one-update}. Then $\rho(c v v \trans + D)$ is non-decreasing with respect to $c$.
\end{lemma}

\begin{proof}
	Let $\epsilon > 0$. Then,
	\begin{equation}
		\begin{aligned}
			\rho((c + \epsilon) v v \trans + D)
			\ = \ \rho(\epsilon v v \trans + (c v v \trans + D))
			\ \geq \ \rho(c v v \trans + D)
		\end{aligned}
	\end{equation}
	by \cref{lem:rank-one-update} because $c v v \trans + D \succeq 0$ by the closure of positive semidefinite matrices under addition.
\end{proof}

Proof of the original result is now quite simple:

\begin{proof}[Proof of \cref{thm:one-dim-implies-multiple-dim}]
	The spectral radius of $\Sigma$ is
	\begin{equation}
		\rho(\Sigma)
		\ = \ \rho(\Var(\beta x + \epsilon))
		\ = \ \rho(a \beta \beta \trans + \Gamma)
	\end{equation}
	Because $\Gamma$ is a covariance matrix, it's symmetric and positive semidefinite and \cref{lem:rank-one-update-increasing} gives us that $\rho(\sigma)$ is non-decreasing in $a$ as desired.
\end{proof}

Strict increase if $\beta$ nontrivially projects onto the first eigenspace of $\Gamma$. This happens on all but a measure zero subset of possible $(\beta, \Gamma)$ pairs.

\begin{proposition}
	If $a = \Var(x) > \rho(\Gamma) = r$, then $r$ is \emph{strictly increasing} with respect to $a$.
\end{proposition}

\begin{corollary}
	There exists some $a > 0$ after which $r$ is strictly increasing.
\end{corollary}

\section{GSS Question Categorization}
\label{sec:gss-questions}

\inlinetodo{include response options}

\inlinetodo{include years questions is asked}

\inlinetodo{move things from misc to police category, rename it ``law enforcement''}

\inlinetodo{move advfront to spending}

\inlinetodo{remove misc section}

\input{./paper/generated/gss_appendix.tex}

\section{Decompositions by Other Characteristics}
\label{sec:other-decompositions}

\foreach \name/\displayName in {
		sex/Sex,
		race/Race,
		religion/Religion,
		geographic_region/Geographic Region,
		type_of_neighborhood/Type of Neighborhood,
		political_ideology/Political Ideology,
		workforce_status/Workforce Status,
		self_or_parent_immigrant/Immigrant Status,
		education/Education,
		age_bucket/Age
	}{
		\begin{figure}
			\centering
			\caption{Polarization Within \displayName}
			\input{./figures/gss/by_group/\name.pgf}
		\end{figure}

		\begin{figure}
			\centering
			\caption{Polarization Trends Within vs. Between \displayName}
			\input{./figures/gss/decompositions/\name.pgf}
		\end{figure}
	}



\end{document}
