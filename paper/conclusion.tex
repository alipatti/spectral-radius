\section{Conclusion}

Introduced and motivated a new measure of polarization in the general public that is broadly applicable to all sorts of data.
Showed several decomposition of measure with intuitive interpretations.

Showed using data from the General Social Survey that polarization in the general public has increased over the last three decades across a wide variety of topics.
This is contrary to general poli sci literature which generally holds that polarization has only increased among elites (e.g. politicians).

These changes are being driven by a variety of factors, both within and across political parties.


\subsection{Future Work}

Explore wider variety of topics.
Would loved to have looked at immigration, but GSS doesn't provide good longitudinal info.

Many subgroup cuts are endogenous (e.g. political party, idology).
Would be nice to use a panel dataset to fix characteristics at t=0.

Use other data sets.

Look at political elites, e.g. congresspeople.
See how elite polarization correlates with constituent polarization.
Use LLM to produce panel of responses to many survey questions.
Decompose into entry/exit vs. changing opinions of people within.

Look at the effect of some sort of exogenous shock on polarization.
Geographic variation.

Take into account clustering rather than just spread \parencite{esteban-ray-1994-measurement-polarization, duclos-esteban-2004-polarization-concepts}.
