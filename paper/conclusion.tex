\section{Conclusion}
\label{sec:conclusion}

In this paper, I develop measure of mass ideological polarization based on the joint distribution of people's opinions.
Concretely, I compute the covariance matrix of responses to various survey questions about views on policy and ideology, and then use the matrix's spectral radius as a measure of polarization.
Intuitively, this captures both the degree to which people disagree on individual issues \emph{and} the between-issue correlation of opinions (\cref{sec:theory}).

Next, I present two decompositions that begin to explore not just whether polarization has increased, but \emph{why}.
The first decomposition partitions polarization into a term capturing the degree to which opinions project onto a single ``us–vs–them'' axis and a term capturing the strength of disagreement in general. The second decomposition partitions changes into within- and between-group components (\cref{sec:decompositions}).

Finally, I apply these methods to three decades of survey data, producing an evolving portrait of ideological polarization in the American public (\cref{sec:gss}).
I find that polarization rises on most topics since the early 1990s (\cref{sec:overall-trends}), driven mostly by increases in general disagreement rather than increased cross-issue opinion correlation (\cref{sec:trace-concentration-decomp}) and by increases in disagreement within groups rather than between them (\cref{sec:party-decomp,sec:gss-other-groups}).
Taken together, this evidence challenges accounts that increasing polarization is due to demographic sorting alone.

\subsection{Future Work}

This paper leaves significant room for future work.
First, one could explore a wider variety of topics using other data sets (for example, the American National Election Studies from \textcite{anes}).
For example, the GSS does not ask any long-running questions about views on immigration---a currently polarizing issue.

Second, the repeated cross-section structure of this paper's data means that the within- versus between-party analysis is effectively conditioning on an endogenous characteristic.
A panel data set would allow one to fix political party (or any other covariate) at some initial $t=0$, then see how polarization evolves within those static groups.
\footnote{The GSS and Pew Research do provide panel survey data sets, although their scope is fairly small and both panels cover a far shorter of a time period than the corresponding repeated cross-section.}

Third, future work could explore the relationship between polarization of political elites (e.g. congresspeople) and the polarization of their constituents.
Do polarized electorates produce polarize politicians?
For example, an extension of the decompositions in \cref{sec:decompositions} could be used to decompose trends in congressional ideological polarization into a component contributed by changing opinions of incumbents and a component contributed by turnover of elected officials.
Are lifetime politicians becoming more polarized in their views, or is congressional polarization a symptom of strongly-opinionated new representatives?

Look at the effect of some sort of exogenous shock on polarization.
For example, \textcite{autor-dorn-hanson-majlesi-2020-importing-political-polarization} find that 

increased polarization in trade-exposed areas---as measured by electoral turnout and campaign contributions.
Does my measure pick up this same phenomenon?\footnote{
	Granular geographic identifiers are available for the GSS \href{https://gss.norc.org/content/dam/gss/get-documentation/pdf/other/ObtainingGSSSensitiveDataFiles.pdf}{via application}.
}

More broadly: can we get geographic variation and then use this to study mechanisms?
Would be statistical issues:
Would have to account for finite sample upward bias in spectral radius estimate.
Could help with this and general power issues by using a Bayesian approach, e.g. fit prior on the eigenvector/change-of-basis diagonalizing matrix or some sort of Markov-field smoother over space.
(Borrowing information from nearby places about the sorts of issues that people are polarized on.)

% Extend our measure to take into account clustering rather than just spread \parencite{esteban-ray-1994-measurement-polarization, duclos-esteban-2004-polarization-concepts}.
% (In bounded data, increased spread inherently creates clustering at the extremes.)

